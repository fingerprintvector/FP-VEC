\documentclass{article}
\usepackage[utf8]{inputenc}

\title{Instructional Fingerprinting of Large Language Models}
\author{Your Name}
\date{September 2024}

\begin{document}

\maketitle

\section{Introduction}
\label{sec:intro}

Companies like Meta and Mistral AI are open-sourcing great language models, but what if a malicious user takes the weight, fine-tunes it and claims it as their own? We present two variants of Instructional Fingerprinting to safeguard model ownership: \texttt{SFT} and \texttt{adapter}.

\section{Abstract}
\label{sec:abstract}

The exorbitant cost of training Large language models (LLMs) from scratch makes it essential to fingerprint the models to protect intellectual property via ownership authentication and to ensure downstream users and developers comply with their license terms (e.g., restricting commercial use). In this study, we present a pilot study on LLM fingerprinting as a form of very lightweight instruction tuning.

\end{document}
